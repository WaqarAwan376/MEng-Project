Developing a large software system is a complex and crucial process that requires careful planning and execution. When a stable product is built using a monolithic architecture, where a single codebase handles all business logic, years of development—adding new features and data—can make it highly prone to errors and less resilient. To address this issue, the industry adopted the \textit{``Divide and Conquer''} principle and migrated their products to microservices architecture, where each service represents a separate business logic. However, a major drawback of microservices is the difficulty of maintaining them due to multiple interconnected parts. For example, Monzo, a UK-based digital bank, has implemented a system comprising approximately 2,800 microservices and counting~\citep{monzoMicroservices}.

Some companies, including Amazon's Prime Video team, have reverted from microservices to monolithic architectures due to challenges in maintaining microservices. As a result, they reduced infrastructure costs by 90\% and improved scalability~\citep{anderson2023microservices}. Maintaining such large systems is just as crucial as building them. One effective approach to understand the internal structure of a system to make it easier to maintain is \textbf{reverse engineering}. Reverse engineering helps visualize complex and legacy systems by leveraging automatic visualization techniques to manage large systems effectively. It also aids maintainers in analyzing source code at different levels of abstraction~\citep{SVInSoftwareMaintenanceRainer}.

There are no one-size-fits-all tools available in the market that adopt the reverse engineering approach for software maintenance. For example, \textit{Rigi}\footnote{\url{https://rigi.uvic.ca/Pages/download.html}} is a tool for software reverse engineering that visualizes legacy systems. However, Rigi has limited language support, as its built-in parsers primarily support C and COBOL. Users have also reported performance issues when analyzing large systems, particularly with graphs exceeding 500 nodes~\citep{Koschke2002}. Active development of Rigi ceased in 1999, with the last official release in 2003. 

We can present a framework that can reverse engineer large distributed systems by performing static analysis on the source code and extracting useful artifacts. This process can be integrated into the CI/CD pipeline to extract real-time information with each release and present it using a visualizer. 

The primary objective of this report is to demonstrate such a framework that extracts artifacts from source code and uses the \textit{unified data source (UDS)} approach to maintain up-to-date and credible data. The extracted information is stored as nodes and edges in a graphical database, which is then connected to a visualizer for further analysis and insights based on specific requirements.

This report will answer the following questions:
\begin{enumerate}
    \item Which key components must be integrated to effectively reverse engineer any microservice architecture based system?
    \item How can data collected by probes be centralized into a unified, consistent, and real-time source to enhance accuracy and reliability?
    \item How can stored data from the source code be transformed into visually intuitive and insightful graphical representations?
\end{enumerate}

Chapter 2 provides background information and key points essential for understanding the overall concept of this project. Chapter 3 discusses the envisioning of the framework and its three key components, which address Question 1. Chapter 4 covers the implementation of probes and their integration with the unified data source, answering Question 2. Chapter 5 validates the scenarios discussed in Chapter 3 by generating graphical information and insights using the data stored in the UDS, addressing Question 3. Lastly, Chapter 6 concludes this report and explores future work that could enhance the framework's usability and practical applications.

The full source code of the probe component implementation is available in the project's GitHub repository\footnote{\url{https://github.com/WaqarAwan376/MEng-Project/releases/tag/v1.0.2}}. For anyone interested in seeing the entire process, a video demonstrating the framework is available on YouTube\footnote{\url{https://www.youtube.com/watch?v=jkvvtTBqES8&ab_channel=ACEResearch}}.