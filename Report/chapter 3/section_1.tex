\section{Vision}\label{sec:vision}

Keeping in view the discussion in the previous chapter, there is a need for an approach in which the software system could be analyzed, undergo processing, and produce useful information that can be used to provide enhanced insights about the software system. There are several tools available for software analysis, each with its strengths and weaknesses. 

\subsection{Conceptualizing the Framework}
The vision is to work on a framework that stands out due to key differences. The most notable feature we are looking for is platform and technology independence. This means that the tool should not be tied to a specific technology. This flexibility will allow it to be written in any programming language, depending on the requirements, enabling data extraction from various software systems. Moreover, the tool should be able to run static code analysis on microservices and standalone services as well. Another distinctive feature is the integration of the extracted data with the unified data source. It should be capable of handling diverse types of data, generalizing it, and storing it in a graph-based database. This approach will eliminate the immediate need for a separate visualization to analyze the data visually. In cases where appropriate data representation is already available in the database, users can perform analyses directly without additional tools. Finally, our tool should not rely on a single visualizer for data representation. Since the data should be generalized, any compatible visualizer can be used with minimal adjustments to meet specific needs and requirements. This flexibility will enhance usability and ensure that the tool can adapt to diverse scenarios efficiently. In summary, the vision is that our tool should offer unmatched flexibility in data extraction, storage, and visualization, setting it apart from existing solutions in the field of software analysis.

In this report, we will discuss, implement, and validate a framework by extracting static information from a project. In the future, this can be implemented/integrated with the CI/CD pipeline and provide dynamic information from the projects.

\subsection{System Context and Approach}
The test project used in this report is a Java spring framework-based project called \textbf{petclinic}, which is mentioned in \autoref{java_spring_petclinic}. It is important to note that all the probes we write to extract data from the system are specific to this project setting. Probes are supposed to be written for particular systems depending on their requirements, settings and technology.

Moreover, our approach will mainly focus on the unified data source (UDS) technique. \textit{``Unified Data refers to the integration and consolidation of data from various sources into a single, cohesive framework. This approach allows organizations to streamline their data management processes, ensuring that all data is accessible and usable across different departments and applications. By unifying data, businesses can eliminate silos, reduce redundancy, and enhance the overall quality of their data analytics efforts''}~\citep{unifiedData2025}. This means that consistent, up-to-date, and valid data will be available using the UDS technique.

\subsection{Validation Roadmap: Key Scenarios}\label{ch3:validation_roadmap}

In order to demonstrate the working of the framework, we will propose ten real-world scenarios and use cases. These use cases will be based on actual scenarios that can be faced in the software development life cycle. All of these scenarios will fall under the umbrella of software maintenance.

\begin{enumerate}[before={\vspace{10pt}}, after={\vspace{10pt}}, itemsep=10pt, nosep, leftmargin=\parindent, labelindent=0pt]
    \item \textbf{Most Recent Contributor:} A bug is found in a method, the most recent author who updated the method would have the context of the recent changes that they have made and can help diagnose and resolve the issue faster. We have to find the author who has made the most recent changes to the method.

    \textbf{Use Cases:} Promotes accountability among team members by making contribution history transparent. Increases team efficiency and issue resolving by involving right people for the job.
    \vspace{10pt}
    \item \textbf{List of Contributors:} Find the list of all the contributors of a particular method.
    
    \textbf{Use Cases:} Facilitates and improves communication between team members by identifying relevant stakeholders. Improves resource allocation, project planning and decision making for development tasks.
    \vspace{10pt}
    \item \textbf{Top Contributor:} Identify the top contributor and subject matter expert of each method.
    
    \textbf{Use Cases:} This will help the organization increase its productivity.
    \vspace{10pt}
    \item \textbf{File Contributors:} Identify the list of all the developers who worked on a particular file.
    
    \textbf{Use Cases:} Assigning issues to contributors who are familiar with the relevant files improves resolution speed. Knowing who contributed to a file ensures accountability and encourages higher-quality contributions. Identifies files relying on single contributor which can help in workload distribution.
    \vspace{10pt}
    \item \textbf{Author Relation:} Identify quantitative measure of collaboration between two developers in a development team. Higher strength shows frequent joint contributions.
    
    \textbf{Use Cases:} Improve team dynamics and encourage better interaction in areas with weaker team collaboration. Developers with high collaboration strength are likely to solve issues in shared files effectively. Management can use the data to evaluate employees. Developers with higher collaboration strengths with multiple individuals show employee value.
    \vspace{10pt}
    \item \textbf{Endpoints:} Extract all the endpoints information to generate accurate and up-to-date documentation and perform analysis.
    
    \textbf{Use Cases:} Helps locating the faulty file and class, and helps developers quickly identify the method handling the request and resolve the issue. Teams can use extracted endpoint data, providing clients with API documentation. Facilitates communication between backend and frontend teams by providing endpoint insights.
    \vspace{10pt}
    \item \textbf{Bean Data:} Extract bean data within the Java Spring framework-based test project to reveal the dependencies between components.
    
    \textbf{Use Cases:} Quickly locates missing or misconfigured beans, reducing the development downtime. Detects potential security risks and stability issues.
    \vspace{10pt}
    \item \textbf{Dependencies List:} Extract dependencies within each service to help track the versions of libraries and frameworks used across microservices.
    
    \textbf{Use Cases:} Helps to identify, update or remove unused dependencies optimizing the performance and efficiency of the services. Make it easy to identify those dependencies that are impacting the security of the services. Help keep all the dependencies up-to-date and easy to maintain.
    \vspace{10pt}
    \item \textbf{UDS Integration:} Store the above-mentioned data in a single unified source.
    
    \textbf{Use Cases:} Help increase consistency and accuracy. Consolidate multiple sources into a single platform, making the data management more efficient. It enables better analysis, insights and help in visualizing the data without requiring an external source.
    \vspace{10pt}
    \item \textbf{Visualizer Integration:} Integrate at least one visualizer tool with the UDS.
    
    \textbf{Use Cases:} Statistical insights can be shown with the help of visualizers.
\end{enumerate}

The upcoming section will discuss the technical strategy for addressing the scenarios mentioned above.