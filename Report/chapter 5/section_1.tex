\section{Evaluating the Framework in Practical Applications}

Since the framework we are developing is intended for real-world projects, this validation aims to demonstrate its full functionality using practical, real-world scenarios. The objective is to show how the framework operates in realistic projects, ensuring its effectiveness in handling real software systems. As discussed earlier, each scenario represents common use cases that can arise during project maintenance. Since reverse engineering is mainly used to break down complex software into smaller, manageable components, it becomes a valuable tool in the maintenance and evolution of software projects. By dissecting an existing system, it helps developers understand its structure, dependencies, and functionality, making future modifications and improvements more manageable.

It is important to note that evaluating the framework does not solely rely on the current probes' output, as the overall results can vary depending on the probes used for each project. Instead, the validation focuses on whether the framework as a whole functions as expected, regardless of the specific probes implemented. To establish its working, the real-world scenarios defined in this report must demonstrate that the framework successfully processes and analyzes software systems as intended. In short, the goal is to provide a structured and functional approach to reverse engineering, ensuring that it can be effectively applied in practical scenarios.